\section{Introduction}
There exists several different approaches that can be used for investigating one of the fundamental building blocks of natural sciences, partial differential equations. The PDEs can appear in almost any field of science, and can give us information about a wide range of topics, such as heat dispersion, wave dynamics and quantum mechanical systems\citep[see][ch. 10]{hjorth-jensen_computational_2015}.

We will be investigating two different approaches to solving partial differential equations(PDEs). The first method being Deep Neural Networks - a multilayer perceptron, and for the second approach will be looking at a finite differences method. For the latter, we will be using Forward Euler. We will first and foremost be investigating solutions with Deep Neural Networks, and use Forward Euler as a method for comparison.

We will begin in section 2 to look at the PDE we are investigating and quickly derive its analytical solution. Then we will quickly go through the finite difference method of Forward Euler to solve the same PDE, and finally go through the basics of a Deep Neural Network(DNN) as applied to PDEs. In section 3 we will briefly sketch out the parameters we will run for, as well as point out where to find relevant code. In section 4 we will go through the results for different combinations of hyper parameters used in the DNN, and compare it with both the analytical results and the Forward Euler results. Then, we will proceed with discussing these results in section 5, and finally in section 6 we will summarize our findings and make appropriate conclusions.