\section{Theory}
A large part of Lattice QCD revolves around the attempt to in trying to extract states, particularly exited states. Problems with this is that they are highly correlated and can be difficult to isolate. Techniques to remedy this exists, in particular the variational method\husk{cite here!!}. We will present an alternate method using \husk{what method?}. But in order to begin, let us first begin by looking at what an exited state is and how a novel quantum mechanical approach can be used to illustrate the problem.

\subsection{Quantum mechanical exited states}
Let us begin with looking at a toy example, namely the path integral as formulated in quantum mechanics. Consider quantum mechanics in one dimension where we start at an initial position eigenstate $\ket{x_i}$ at time $t_i$ and end up at a final position eigenstate $\bra{x_f}$ at time $i_f$,
\begin{align}
    \bra{x_f} \e{-\hat{H}(t_f - t_i)}\ket{x_i} = \int \mathcal{D} x(t) \e{-S[x]}.
    \label{eq:qm-path-integral}
\end{align}
The $\mathcal{D}$ is indicating the \textit{path integral}, which means that we are integrating over all possible paths\cite{peskin_introduction_1995,shankar_principles_1994}. Each of these paths are weighted by some (Wick rotated) action $S[x]$, which is a functional of the path.

Our goal is to extract exited states from this function \eqref{eq:qm-path-integral}. In order to do that, we have to rewrite 



\subsection{A brief primer on Lattice QCD}

\subsection{Correlators}

\subsection{Variational Method}

\subsection{Extracting states with machine learning}
